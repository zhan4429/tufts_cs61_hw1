\header{HW 1}[\due Mon,  9 Sep 2024] 

\textbf{\color{blue} Note: For full credit you need to explain your answers, unless the problem is clearly an exception.  For example, in part III problems 3.1 and 3.3 you should explain your answers, whereas in 4.1a-d it is clearly asking you to simply restate the statement in the correct form, no additional explanation required.  This is true for all homeworks and exams in this class.}

\bigskip
\bigskip

\Part{I} Reading and Resources

\begin{enumerate}

\item (4 pts) Read Scheinerman \textbf{3rd ed.}, sections 1--4.  (These sections are available on Piazza under ``Resources".)  For each section, describe one thing you learned and/or enjoyed reading.  Usually a sentence or two per section is fine, but we love it when you write more.  

\item (1 pt) List all resources you end up using to help you solve the homework problems below (e.g., people, TAs, websites)

\end{enumerate}

\bigskip
\hrule
\bigskip

\Part{II} 
From Raymond Smullyan's \textit{What is the Name of this Book?}

{\em When Alice entered the Forest of Forgetfulness, she did not forget everything, only certain things. She often forgot her name, and the one thing she was most likely to forget was the day of the week. Now the Lion and the Unicorn were frequent visitors to the forest. The Lion lies on Mondays, Tuesday, and Wednesdays, and tells the truth the other days of the week; the Unicorn, on the other hand, lies on Thursday, Fridays and Saturdays, but tells the truth on the other days of the week.\/}

\soln{Put any comments or parts of the solution that are common to all the subparts here.}

\begin{enumerate}

\item One day, Alice met the Lion and the Unicorn resting under a tree. They made the following statements: 
\begin{description}
\item[Lion:] Yesterday was one of my lying days.
\item[Unicorn:] Yesterday was one of my lying days too.
\end{description} 
What day(s) of the week could it be? Explain your answer carefully. 

\soln{Put your solution here.}


\item On another occasion, Alice met the Lion alone. He made the following two statements: 
\begin{enumerate} 
\item I lied yesterday.
\item I will lie again two days after tomorrow. 
\end{enumerate} 
What day(s) of the week could it be? Again, explain your answer carefully. 

\soln{Put your solution here.}


\item On what days of the week is it possible for the Lion to make both of the following two statements?  (again, explain.)
\begin{enumerate}
\item I lied yesterday.
\item I will lie tomorrow. 
\end{enumerate} 

\soln{Put your solution here.}


\item On what days of the week is it possible for the Lion to make the following single statement (again, explain): ``I lied yesterday and I will lie tomorrow''?

\soln{Put your solution here.}


\bigskip
\hrule
\bigskip

\end{enumerate}

\Part{III} Do the following problems from \thetextbook. 

Note: We are most interested in the clarity of your explanations, not just the correct answer.
\begin{itemize}
\item[\bfseries\S 3:] 1, 3 

\item[\bfseries\S 4:] 1a-d, 3, 4, 5 

\item[$\bullet$] Final question (1 pt): How much time did you spend on this homework, in hours?

\end{itemize}

